\documentclass[a4paper, 12pt, answers]{exam}
%\documentclass[a4paper, 12pt]{exam}

%\printanswers

\usepackage{ctex}
%\usepackage{tabulary}
\usepackage{graphicx}

\pagestyle{headandfoot}

\firstpageheadrule
\firstpageheader{并行计算机体系结构}{期末考试}{2019年6月11日}
\runningheadrule
\runningheader{并行计算机体系结构}{}{期末考试}
\firstpagefooter{}{}{}
\runningfooter{}{第 \thepage\ 页,共 \numpages 页}{}
\runningfootrule

\begin{document}

\begin{center}

\fbox{\fbox{\parbox{5.5in}{
\vspace{0.1in}
\centering
\large 并行计算机体系结构期末考试\\2019年6月11日
\vspace{0.1in}
%Answer the questions in the spaces provided after the
%question sheets. If you run out of room for the answers,
%continue on the back of the page.
}}}

\end{center}

\vspace{0.1in}
\noindent
\makebox[\textwidth]{姓名:\enspace\hrulefill}

\vspace{0.2in}
\noindent
\makebox[\textwidth]{学号:\enspace\hrulefill}

\vspace{0.2in}

\centering

	\hqword{题号}
	\hpword{分数}
	%\hbpword{得分}
	\hsword{得分}
	\htword{总分}

	\addpoints

\gradetable[h][questions]

\vspace{0.5in}

%\vspace{0.2in}
%\centering\large{题目}
%\vspace{0.2in}

\pointname{分}

\begin{questions}

\question[15]
请简要介绍计算机体系结构的Flynn分类并举例。

\begin{solution}[12cm]
%{ } \\
	Flynn分类的示意图如下:(7分) \\
	\includegraphics[width=0.9\textwidth]{FlynnsTaxonomy}

	\begin{itemize}
		\item{SISD: 单核处理器 (2分)}
		\item{SIMD: GPU流处理器,向量处理器 (2分)}
		\item{MISD: 多模冗余处理器 (2分)}
		\item{MIMD: 多核处理器,集群系统 (2分)}
	\end{itemize}

\end{solution}


\question[20]
基于监听的MESI协议是一种经典的Cache一致性协议。请简要介绍MESI协议中各个状态的含义,并说明从I到S或E状态,以及从M到I或S状态转换的规则。 \\

%\includegraphics[width=\textwidth]{mesi}

\begin{solution}[12cm]
%{ } \\
	MESI状态含义是: \\
	\begin{tabular}{| c | p{13cm} |}
		\hline
		M & 这行数据有效,数据被修改了,和内存中的数据不一致,数据只存在于本Cache中。(2分) \\ \hline
		E & 这行数据有效,数据和内存中的数据一致,数据只存在于本Cache中。(2分) \\ \hline
		S & 这行数据有效,数据和内存中的数据一致,数据存在于很多Cache中。(2分) \\ \hline
		I & 这行数据无效。(2分) \\ \hline
	\end{tabular}

	从I到S状态转换是本地CPU发起读操作,且有其它CPU声明S状态时。(3分)

	从I到E状态转换是本地CPU发起读操作,且没有其它CPU声明S状态时。(3分)

	从M到I状态转换是发现总线上有其它CPU发出BusRdX,即为了写的总线读,本地CPU写回该行,并转到I状态。(3分)

	从M到S状态转换是发现总线上有其它CPU发出BusRd,即总线读,本地CPU写回该行,并转到S状态。(3分)

\end{solution}


\question[20]
GPU已经成为计算加速的常用部件,其加速计算的过程通常有3个阶段:(1)从主存拷贝输入数据到显存;(2)启动GPU上的kernel函数进行大规模数据并行计算,产生输出数据;(3)拷贝输出数据到主存。假设某个计算程序GPU上的kernel函数平均每个双精度浮点运算访问显存0.1字节,消耗输入数据0.01字节,产生输出数据0.001字节。如果该计算程序使用Nvidia的Tesla V100带NVLINK接口(技术规格见下表)来加速,那么性能瓶颈将会出现在那个阶段?如果采用PCIE接口的话,瓶颈又会出现在哪里?

	\begin{table}[h]
\centering
\begin{tabular}{| c | l |}
	\hline
	双精度处理性能 & 7.8T(NVLINK), 7T(PCIE) FLOPS \\ \hline
	NVLINK带宽 & 300GB/s \\ \hline
	PCIE带宽 & 32GB/s \\ \hline
	显存带宽 & 900GB/s \\ \hline
\end{tabular}
	\end{table}
	\raggedright

\begin{solution}[10cm]
%{ } \\
%\raggedright 假设单核串行执行时间为T。\\
	瓶颈可以认为是计算过程中达到了最大带宽或处理性能的阶段。(2分)\\
	使用NVLINK接口: \\
	如果处理性能饱和,需要的显存带宽为 7.8T * 0.1 = 0.78 TB/s = 780 GB/s ,小于显存带宽(2分)\\
	需要的输入带宽为 7.8T * 0.01 = 0.078 TB/s = 78 GB/s ,小于NVLINK接口带宽(2分)\\
	需要的输出带宽为 7.8T * 0.001 = 0.0078 TB/s = 7.8 GB/s ,小于NVLINK接口带宽(2分)\\
	因此瓶颈在GPU双精度处理阶段。(3分)
	
	使用PCIE接口: \\
	如果处理性能饱和,需要的显存带宽为 7T * 0.1 = 0.7 TB/s = 700 GB/s ,小于显存带宽(2分) \\
	需要的输入带宽为 7.8T * 0.01 = 0.07 TB/s = 70 GB/s ,大于PCIE接口带宽(2分)\\
	需要的输出带宽为 7.8T * 0.001 = 0.0078 TB/s = 7 GB/s ,小于PCIE接口带宽(2分)\\
	因此瓶颈在输入数据阶段。(3分)
	

\end{solution}


\question[20]
Infiniband互连网络由于其高带宽低延迟的特性,在超级计算机中得到了广泛应用。请问相对于以太网,Infiniband互连网络的主要靠什么做到高带宽低延迟的?请简要回答。 \\

\begin{solution}[10cm]
%{ } \\
%网络拓扑结构特性表如下所示。
%\begin{tabular}{| c | l |}
%	\hline
%	拓扑结构 & 网络直径 & 对剖宽度 & 成本 & 延迟 & 汇聚带宽 \\ \hline
%	总线 & 1 & 1 & O(n+w) & O(1) & O(w) \\ \hline
%	环 & n/2 & 2 & O(n+w) & O(d) & O(w) \\ \hline
%	交叉开关 & 1 & sqrt(n) & O(n*n*w) & O(1) & O(n*w) \\ \hline
%	树 & log(n) & 1 & O(n) & O(logn) & O(nlogn*w) \\ \hline
%	Omega网络 & logn & logn & O(nlogn*w) & O(logn) & O(nlogn) \\ \hline
%	超立方 & n & n/2 & O(nlogn/2) & O(d) & O(nlogn) \\ \hline
%	网格 & 2*(sqrt(n)-1) & sqrt(n) & O(n) & O(sqrt(n)) & O(sqrt(n)) \\ \hline
%	环形网格 & sqrt(n)-1 & 2*sqrt(n)& O(n) & O(sqrt(n)) & O(sqrt(n)) \\ \hline
%\end{tabular}
	Infiniband互连网络主要优势有两点:(1)采用切通(cut-through)式交换,比以太网存储转发式交换效率高,延迟小,见下图1;(10分)(2)采用RDMA技术,减少消息包在不同层缓存,进一步降低延迟,见下图2。(10分)

\centering
\includegraphics[width=\textwidth]{storeforewardvscutthrough}


\includegraphics[width=\textwidth]{rdma}
\raggedright

\end{solution}


\question[25]
下图是当前排名第一和第二的超算Summit和Sierra半个计算结点结构图。这两个超算都采用了3层k度胖树作为互连拓扑,利用100Gbps的Infiniband交换机分别连接了4608个和4320个计算结点,每个结点2个22核Power9微处理器,6个或4个英伟达Volta~V100s计算加速卡。(1)请估算k的值;(2)请估算Summit和Sierra的峰值性能(可只考虑GPU性能);(3)请分析Summit和Sierra的瓶颈所在。 \\

\centering
\includegraphics[width=0.9\textwidth]{summitandsierrahalfnode}
\raggedright

\begin{solution}[10cm]
%{ } \\
	\raggedright
	胖树k值: \\
	k度胖树可以连接 k3/4 个结点,故 Summit的 k 值 = 4068*4 开3次方,大约27,取偶数得28。Sierra的k值为26。 (9分)

%	胖树带宽: \\
%	Summit使用Infiniband的EDR连接,单端口可达100Gbps,故叶子层 100 Gbps,第一层交换带宽 27/2*100Gbps = 1.35Tbps,第二层应该与第一层一样,第三层带宽为 27*100Gbps = 2.7 Tbps (8分)

	Summit峰值: \\
	V100性能7 TFLOPS,单个结点6个加速卡总共 42 TFLOPS,总峰值性能 = 4608 * 42 TFLOPS = 194 PFLOPS。(6分)

	Sierra峰值: \\
	V100性能7 TFLOPS,单个结点4个加速卡总共 28 TFLOPS,总峰值性能 = 4320 * 28 TFLOPS = 121 PFLOPS。(4分)

	瓶颈:显然瓶颈在通信和I/O带宽。(6分)



\end{solution}



\end{questions}

\end{document}
